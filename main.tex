\documentclass{article}
\usepackage{fancyhdr}
\usepackage{tabularx}
\usepackage{geometry}
\usepackage{lipsum}
\usepackage{amssymb}
\usepackage{venndiagram}
\usepackage{subcaption}
\usepackage{wrapfig}
\usepackage{multicol}
\usepackage{float}
\usepackage{mathpazo}
\usepackage{amsmath, amssymb}
\usepackage[utf8]{inputenc}
\usepackage[T1]{fontenc}
\usepackage{geometry}
\usepackage{graphicx}
\usepackage{xcolor}

\title{\huge Einleitung in die Stochastik}
\author{Dein Name}
\date{\today}

\begin{document}

\begin{titlepage}
    \begin{center}
        \vspace*{2cm}
        
        %\includegraphics[width=0.5\textwidth]{university_logo} % Hier das Logo deiner Universität einfügen
        
        %\vspace{2cm}
        
        %{\LARGE \textbf{Fakultät für Mathematik}} \\
        %\vspace{0.5cm}
        %{\Large \textbf{Institut für Stochastik}}
        
        %\vspace{3cm}
        
        \rule{\textwidth}{0.4pt} \\
        \vspace{0.5cm}
        {\huge \textbf{Einleitung in die Stochastik}} \\
        \vspace{0.5cm}
        \rule{\textwidth}{0.4pt}
        
        \vspace{3cm}
        
        %{\Large \textbf{Bachelorarbeit/Masterarbeit/ Seminararbeit}} \\
        
        %\vspace{1.5cm}
        
        %{\large \textbf{Vorgelegt von}} \\
        %{\Large \textbf{Dein Name}} \\
        %{\large \textbf{Matrikelnummer: XXXXXXX}}
        
        \vfill
        
        {\large \textbf{Datum:} \today}
        
    \end{center}
\end{titlepage}


% Seitenränder einstellen
%\geometry{a4paper, left=2.5cm, right=2.5cm, top=2.5cm, bottom=2.5cm}

% Header Definition inieren
\pagestyle{fancy}
\fancyhf{}
\renewcommand{\headrulewidth}{1pt}
%\lhead{Lukas Dzielski}
\rhead{Ruprecht-Karls-Universität Heidelberg \\ MaFIn}
\lfoot{}
\rfoot{\thepage}

\section*{Grundlagen der Wahrscheinlichkeitstheorie}

\paragraph*{Ziel:} Mathematisches Modell, welches die intuitive Idee "Wahrscheinlichkeit"(="relative Häufigkeit") beschreibt.
\paragraph*{Anwendung:} 
\begin{itemize}
    \item Modellierung von Komplecxen Systemen (Aktienmarkt, Verkehr, Anfragen bei einem Server)
    \item Randomisierte Algorithmen (Las-Vegas Algo., Modelierungs Algo. Primzahlentest)
    \item Performance-Annalyse von Algorithmen (Durchschnittliche Laufzeit bei zufälligen Input)
\end{itemize}

\subsection*{Informelle Einführung:}
\paragraph*{Experiment:} Würfeln mit zwei Würfeln A und b
\paragraph*{Ausgang:} Ein Paar von Augenzahlen je zwischen 1 und 6 z.B. A zeigt 1 und B zeigt 4 \(\leadsto\) (1,4) Elementarereignis
Ausgangsraum: \(6 \cdot 6 = 36\) Möglichekeiten.\\
\\ \(\Omega = \{\) 
\\ \((1,1), (1,2), (1,3), \cdots, (1,6)\)
\\ \((2,1), (2,2), (2,3), \cdots, (2,6)\)
\\ \(\cdots\}\)

\subsection*{Sinnvolle Modellierung: }
\begin{itemize}
    \item[(i)] Eine Aussage des Experiments sollte genau einem \(w \in \Omega\) entsprechen.
    \item[(ii)] \(\Omega\) sollte keine unmöglichen Ausgänge enthalten.
\end{itemize}

\subsection*{Wahrscheinlichkeiten}
\paragraph*{Zuweisungen:} Zu jedem \(w \in \Omega\) \(p(w) \in [0, 1]\) zuweisen sodass \(\Sigma_{w \in \Omega} p(w) = 1\)
Idee: Ein Elementarereignis muss auftreten.
Hier: "faire Würfel" => gibt keinen Grund \(p(1,3) > p(5,6)\) 
Also \(p(1,1) = \cdots = p(6,6) = \frac{1}{36}\)
Möchten umfassende Ereignisse beschreiben z.B. die Summe der Augenzahl ist 3.
\(A =\{(1,2), (2,1)\} \subseteq \Omega\)
intuitiv: \(P(A) = p(1,2) + p(2,1) = 2 \cdot \frac{1}{36} = \frac{1}{18}\)

\subsection*{Formale Einführung (Kolmogorov, 1933)}
\paragraph*{Definition:} Ein \textbf{diskreter Wahrscheinlichkeitsraum} ist ein Paar \((\Omega, P)\) wobei: \(\Omega\) eine endliche oder eine abzählbar unendliche Menge ist (\textbf{Elementarereignise}) und \(P \cdot \mathcal{P} \rightarrow)\mathbb{R}\) ein \textbf{Wahrscheinlichkeitsmaß}, d.h (Wahrscheinlichkeitsverteilung)
\begin{itemize}
    \item [(i)] \(P(A) \geq 0 \forall A \in \mathcal{P}(A)\) Nichtnegativitätsgesetz
    \item [(ii)] \(P(\Omega) = 1\) (Normierung)
    \item  [(iii)] Für jede Folge \((A_i)_{i \in \mathbb{N}}\), \(A_i \in \mathcal{P}(\Omega)\), sodass \(A_i \cap A_j = \varnothing \forall i \not = j\) gilt \(P(\cup_{i \in \mathbb{N}} A_i) = \Sigma_{i \in \mathbb{N}} P(A_i)\)
\end{itemize}

\paragraph*{Insbesondere:} \(A, B \in \mathcal{P}(\Omega), A \cap B \not = \varnothing\) \(P(A \cup B) = P(A) + P(B)\)
\paragraph{Eigenschaften:} \((\Omega, P)\) diskreter W-raum \(A, B \subseteq \Omega\) (Ereignisse)
\begin{itemize}
    \item [(a)] \(P(\varnothing) = 0\) \paragraph*{Denn: } \(1 =_{(ii)} P(\Omega) = P(\Omega \cup \varnothing) =_{(iii)} P(\Omega) + P(\varnothing) =_{(ii)} 1 + P(\varnothing) => P(\varnothing) = 0 \)
    \item [(b)] \(P(A^c)(=\Omega/A) = 1 - P(A)\) \paragraph*{Denn:} \(P(A) + P(A^c) =_{(iii)} P(A \cup A^c) =_{(ii)} P(\Omega) = 1\)
    \item [(c)] \(A \subseteq B => P(A) \leq P(B)\) \paragraph*{Denn: } \(P(B) = P(A \cup (B/A)) =_{(iii)} P(A)+P(B/A) \leq_{(i)} P(A)\)
    \item [(d)] \(P(A \cup B) = P((B/A) \cup (A\cap B)) = P(B/A) + P(A\cap B)\) \paragraph*{Denn:} \(P(B) = P((B/A) \cup (A \cap B)) = P(B/A) + P(A \cap B) => P(B) - P(A\cap B)\) \\ \(P(A\cup B) = P(A \cup (B/A) =_{(iii)}) P(A) + P(B/A) =_{(*)} P(A) + P(B) - P(A \cap B)\)
    \item [(e)] \(P(A \cup B) \leq P(A) + P(B)\) (folgt aus d + (i))
\end{itemize}

\subsection*{Gesetz disjunktiver Wahrscheinlichkeiten}
\((\Omega, P)\) diskreter W-Raum, \(A = \{w_1, w_2, \cdots\} \subseteq \Omega\) \\ Ereignis \(P(A) =_{(iii)} P(\{w_1\}) + P(\{w_2\}+ \cdots) = \Sigma_{i \in \mathbb{N} P(\{w_i\})}\)
\paragraph*{Definition : } Eine Funktion \(p : \Omega \to [0, 1] sodass \Sigma_{w \in \Omega} p(w) = 1\) heißt \textbf{Wahrscheinlichkeitsfunktion}.
\paragraph*{Beispiel:} Würfel mit zwei Würfeln \(\Omega = \{1, 2, 3, 4, 5, 6\}^2, p(w) = \frac{1}{36} \forall w \in \Omega\) \\ \(P(A) = \Sigma_{w \in A} p(w) = \Sigma_{w \in A} \frac{1}{36} = \frac{\vert A \vert}{36} \forall A \subseteq \Omega\)
\begin{itemize}
    \item [(i)] \(P(A) = \frac{\vert A \vert}{36} \leq 0\)
    \item [(ii)] \(P(\Omega) = \frac{\vert \Omega \vert}{36} = 1\) \\ \(A \cap B  = \varnothing P(A \cup B) = \frac{\vert A \cup B \vert}{36} = \frac{\vert A \vert + \vert B \vert}{36} = \frac{\vert A \vert }{36} + \frac{\vert B \vert }{36} = P(A) + P(B)\)
\end{itemize}

Beschreibe das Ereignis, dass mind. ein Würfel eine 1 zegt \(A = \{(1,1), \cdots, (1,6), (6,1) \cdots, (2,1)\}\) \(P(A = \frac{11}{36})\)

\paragraph*{Definition : } Ein \textbf{endlicher} diskreter W-raum \((\Omega, P)\) heißt \textbf{gleichverteilt}, falls \(P(A) = \frac{\vert A \vert}{\vert \Omega \vert} \forall A \subseteq \Omega\) Wir können die gleichverteilung auf jeder endlcihen Menge \(\Omega\) Definition iniert und erhalten ein W-Maß.

\subsection*{Beispiel: } 
\begin{itemize}
    \item [(a)]Würfer mit 2 roten Seiten und 4 blauen Seiten.
    \(\Omega = \{rot, blau\}, p(rot) = \frac{2}{6} = \frac{1}{3}, p(blau) = \frac{2}{3}\) 
    \item[(b)] \(\Omega = \{a, b, c, d\}\) (nicht gleichverteilt) \(p : \Omega \to [0,1]\) mit \(\Sigma_ w\in \Omega p(w) = 1\) \textbf{hier tabelle einfügen a = 1/10, b= 1/5, c = 1/2, d = ?} \(P(A) = \Sigma_{w\in A} p(w)\)
    \item[(c)] abzählbar unendlich \(\Omega =\mathbb{N}, p(n) = (\frac{1}{2})^n\) \(P(\Omega) = \Sigma^{unendlich}_{n = 1} (\frac{1}{2})^n = \frac{1}{1-\frac{1}{2}} - 1 = 1\) Viele Experimente bestehen aus der Wiederholung desselben grundlegenden Experiments. \paragraph*{Definition :} Seien \(\Omega_1, \cdots, \Omega_n\)diskrete W-Räume mit W-funktionen \(p_1, \cdots, p_m\). Dann ist der \textbf{Produktraum} von \(\Omega_1, \cdots, \Omega_n\), das karteische Proukt \(\Omega_1 \times \cdots \times \Omega_n = \{(w_1, \cdots, w_n), w_i\in \Omega_i, 1 \leq i \leq n\}\) zusammen mit dem W-Maß Definition iniert durch die W-funktionen Definition iniert durch 
    \[ 
        p((w_1, \cdots, w_n)) = P_1(w_1) \cdot p_n(w_n) = \Pi^n_{i = 1 p_i(w_i)}
    \]
    \paragraph*{Beispiel:} Mit zwei Würfeln. Sei \(\Omega' = \{1, 2, 3, 4, 5, 6\}\) mit \(p' = \frac{1}{6} \forall a \in \Omega'\) der W-Raum der den Wurf von einem Würfel beschreibt, dann \(\Omega = \Omega' \times \Omega'\) \\ \(p((a,b)) = p'(a) \cdot p'(b) = \frac{1}{6} \cdot \frac{1}{6} = \frac{1}{36}\)
\end{itemize}
\paragraph*{Satz:} Der Produkraum \(\Omega_1 \times \Omega \times \cdots \Omega_n\) ist ein diskreter W-Raum.

\paragraph*{Beweis:} \(1 \geq P((w_1, \cdots, w_n)) = p_1(w_1)\cdots p_n(w_n) \geq 0\) \\
\(p: \Omega_1 \times \cdots \Omega_n \to [0, 1]\) \\
\[
    \Sigma_{(w_1, \cdots, w_n) \in \Omega_1 \times \cdots \times \Omega_n} p((w_1, \cdots, w_n)) 
\]
\[  = \Sigma_{w_1 \in \Omega_1} \Sigma_{w_2 \in \Omega_2} \cdots      \Sigma_{w_n \in \Omega_n} p_1(w_1 \cdots p_n(w_n)) 
\]
\[ = (\Sigma_{w_1 \in \Omega_1} p_1(w_1))(\Sigma_{w_2 \in \Omega_2} p_2(w_2)) \cdots (\Sigma_{w_n \in \Omega_n} p_n(w_n))
\]

\subsection*{Unabhängigkeit}
Diskreter W-Raum \((\Omega, P)\) Ereignis \(A \subseteq \Omega\) Formalisierung Für die Idee, dass zwei Ereignisse A und B ich nicht beeinflussen. 
\paragraph*{Def:} Zwei Ereignisse \(A,B \subseteq \Omega\) heißen \textbf{Unabhängig}, falls \(P(A \cap B) = P(A) P(B)\)
\paragraph*{Bsp:} Wurf eines Würfels \(\leadsto \Omega = \{1,2,3,4,5,6\}, A =\{2,4,6\}, B = \{1,2\}, C=\{2,3,5\}\) \(P(A \cap B) = P(\{2\}) = \frac{1}{6}\) \(P(A) \cdot P(B) = \frac{1}{2} \cdot \frac{1}{3} = \frac{1}{6}\) => A und B Unabhängig \(P(A \cap B ) = \frac{1}{6}\) \(P(A) \cdot P(C) = \frac{1}{2} \cdot \frac{1}{2} = \frac{1}{4}\) => A und C sind nicht Unabhängig.

\paragraph*{Def:} Eine Menge \(\{A_1, \cdots, A_2\}\) von Ereignissen ist \textbf{Unabhängig} falls \(P(\bigcap^n_{i=1} A_i) = \Pi^n_{i = 1}P(\bigwedge,)\)
\paragraph*{Im Bsp:} Angenommen wir wissen, dass A eingetreten ist. Is es dann meht oder weniger wahrscheinlich, dass auch C eingetreten ist?
\(P(C \vert A) = \frac{1}{3}, P(C) = \frac{1}{2}\)
\paragraph*{Def:} Die \textbf{bedingte Wahrscheinlichkeit} \(P(B \vert A)\) vin B unter der Annahme, dass A eingetreten ist, ist \(P(B \vert A) = \frac{P(B \cap A)}{P(A)}\), \(P(A) > 0\)

\paragraph*{Angenommen} \(A \subseteq B\) \(P(B \vert A) = \frac{P(B \cap A)}{P(A)} = \frac{P(A)}{P(A)} = 1\)

\paragraph*{Beobachtung: } Seien A und B unabhängig. Dann \(P(B \vert A) = \frac{P(A \cap B)}{P(A)} = \frac{P(A) \cdot P(B)}{P(A)} = P(B)\) (A liefert keine zusätzliche Information ob B eingetreten ist!)

\paragraph*{Bsp:} Würfer mit zwei Würfeln B: Ereignis, dass die Summe der Augenzahl = p ist \\ A: Ereignis, dass der 1. Wurf = 5 ist.\\ \(B = \{(4,5), (5,4), (3,6), (6,3)\}\)\\ \(A = \{(5,1), (5,2), ...\}\)\\ \(P(B \vert A) = \frac{P(A \cap B)}{P(A)} = \frac{\frac{1}{36}}{\frac{1}{6}} = \frac{1}{6}\)\\ \(P(B) = \frac{4}{??} = \frac{1}{?}\)

\subsection*{Gesetz der totalen Wahrscheinlichkeit} 

\(\Cup\) soll später getauscht werden mit punkt drin symbol!!!!\\

\(\Omega = \Cup^n_{i = 1} E_i, E_i \cap E_j = \varnothing i \not = j, A \subseteq \Omega\) 
\paragraph*{Dann:} \(P(A) = Sigma^n_{i = 1} P(A \vert E_i) \cdot P(E_i)\)

\paragraph*{Beweis:} \(A = \Cup^n_{i = 1} (A \cap E_i), P(A\cap E_i) = P(A \vert E_i)\)

\[hier text einfügen noch was fehlt...\]

Also \(P(A) =_{3. Axiom} \Sigma\)

\subsection*{Satz von Bayes:} \(A,B \subseteq \Omega, P(A), P(B) \not = 0\) \\ Dann \(P(B \vert A) = P(A \vert B) \cdot \frac{P(B)}{P(A)}\) 
\paragraph*{Beweis:} \(P(A \vert B) = \frac{P(A \cap B)}{P(B)}\), \(P(B \vert A) = \frac{P(A \cap B)}{P(A)}\) \\ =>\(P(A \vert B) \cdot P(B) = P(B \vert A) = P(A)\) \(\Box\)

\section*{Diskrete Zufallszahlen}
Oft möchte man Ereignissen einen nummerischen Wert zuordnen (Summe der Augenzahlen, Gewinn). 

\paragraph*{Def:} Eine \textbf{diskrete Zufallszahlen} X ist eine Funktion \(X: \Omega \to \mathbb{R}\) auf einen diskreten W-Raum \((\Omega, P)\)

\paragraph*{Bem:} X ist keine "Variable" im üblichen Sinne, sondern eine Funktion. \\

Wir können einen W-Raum auf den Werten (Bilderraum) von X definieren. 

\paragraph*{Satz} Sei X diskreter zufallsvariable auf \((\Omega, P)\). \\ \(\Omega_{X} = im(X) = \{x \in \mathbb{R} \vert E??=_w \in \Omega : X(w) = x\}\) \(Px^{(x)} = P(X^{-1}(x)) P= (\{w \in \Omega \vert X(w) = x \})\) deiniert eine W-funktion auf \(\Omega_X\).

\[Hier wichtig checke ob X richtig groß oder klein ist!!!\]

\paragraph*{Beweis:} \(0 \leq P_X(x) \leq 1 \quad \forall Ax\in \Omega_X \) bei def (hier photos checken!!?) \\ \(\Omega = \Cap_{x \in \Omega_X} X^{-1}(x)\), \(\sigma_{x \in \Omega_X} P_X(x) = \Sigma_{x \in \Omega_X}\) <- Das ist einfach komplet falsch ich muss das noch mal über arbeiten??

\paragraph*{Def:} Das Wahrscheinlichkeitsmaß definiert auf \(\Omega_X = im(X)\) mittels \(P_X(x) = P(X = x)\) heißt die \textbf{Verteilung} von X.

\paragraph*{Bsp:} Bei einem Wurf eines fairen Würfels werden den Agenzahlen folgende Gewinne zugewiesen. 

\[hier tabelle einfügen\]

definiert Zufallsvariable \(X: \{1,2,3,4,5,6\} \to \mathbb{R} \quad P(X = 1) = P(X^{-1}(1)) = P(\{w \in \Omega \vert X(w) = 1\}) = P(\{1,2,3\}) = \frac{3}{6} = \frac{1}{2}\) \(\Omega_X = \{1,2,3\}\)

\[hier nochmal tabelle :'(\]
(Verteilung)

\subsection*{Binomialverteilung:} mit Parametern \(n \in \mathbb{N}\), \(p \in [0,1]\). ist definiert ais \(\Omega_X = \{0,1, \cdots, n\}\) durch die W-funktion \(b(k, n ,p) = \binom{n}{k} p^k (1-p)^{n-k}\)

\subsection*{Bernoulli-Experiment:} Versucgh mit zwei Ausgängen \(\{1, 0\}\) \(P(1) = p, P(o) = q = 1-p\).

\subsection*{Bernouilli-Prozess:} Die Folge von Wiederholungen dasselben Bernoulli-Experiments. Ein Prozess mit n Versuchen kann mittels \(\Omega^n\) beschreiben werden. \((0,1,0,0,1,1, \cdots)\) \\Die Wahrscheinlichkeit für k Erfolge ist gegeben durch \(b(k,n,p)\). 

\subsection*{Hypergeometrische Verteilung}
parameter \(K,n,r,m \in \mathbb{Z}_{\geq 0}\) mit \(0 \leq k \leq n \leq m\), \(k \leq r\). Defniniert durch \(h(K, n,r,m) = \frac{\binom{r}{k} \binom{m-r}{n-k}}{ \binom{m}{n}}\)

\paragraph*{Interpretation:} Urne mit r roten Kugeln, m-r blaue Ziehen n Kugeln ohne zurückzlegen (auf einmal) \\ \(\Omega\) = Jede Menge von n Kugeln \\ \(P\) = gleichverteilt \\ \(h(k; n,r,m) = P\) was kommt hier rein????

\paragraph*{Def:} Der \textbf{Erwartungswert} einer diskreter Zufallsvariable X ist (Was für n komisches E ist das denn bitt??) \(E[x] = \Sigma_{x \in \Omega_X} p_X (x) X = \Sigma_{x \in \Omega_X} P(X=x)\cdot x\) und wird oft mit mü (wie mükrosekunde) bezeichnet.

\paragraph*{Bsp:} Beim Würfeln einer geraden Augenzahl verliert man ???? Beim Würfeln einer ungeraden gewinnt man die Augenzahl in ???. \\ \(\Omega_X = \{-3,1,3,5\}\) \[hier noch tabelle\] \(??? = ????\)

\paragraph*{Satz:} X diskreter Zufallsvariable über \((\Omega, P)\). Dann \(??? = \Sigma_{w \in \Omega}X(w) p(w)\)

\paragraph*{Beweis:} \(??? = \Sigma_{x \in \Omega_X} Xp_x(x) = \Sigma_{x \in \Omega_X} x \cdot P(X^-1)(x)) = \Sigma_{x \in im(X)} X \)
\end{document}



